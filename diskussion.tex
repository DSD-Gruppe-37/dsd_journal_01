\section{Diskussion}

\subsection{Half-adder}

Alle tre kodestilarter producerer efter syntese det samme logiske kredsløb som
på chippen bruger 2 logiske elementer. Funktionelt- og timingmæssigt opfører
simulationerne for alle tre sig ens. Dette giver mening da alle tre half-addere
er ens efter syntese.

\subsection{Full-adder}

I \figref{FullAdderTechnologyMap} ses det hvordan \emph{synthesizeren} har mappet Full-adderen ned til blot 2 logiske elementer, istedet for de 5 som den var designet med.

\dsdfig{FullAdderTechnologyMap}{9cm}{Technology Map View af Full-Adderen (post-mapping).}{tpb}

\subsection{4-bit adder}

\dsdfig{Technologymapview}{9cm}{Technology Map View (post-mapping).}{tpb}

På \figref{Technologymapview} ses det hvordan \emph{synthesizeren} får mappet funktionerne ned på den aktuelle chip.

\subsection{Pipeline og command line tools}

Når et program bygges og flashes til DE2-boardet gennemgår det en række
forskellige steps (en build pipeline).

\emph{Synthesis} oversætter koden til et logisk kredsløb af almindelige logiske elementer (også kaldet
RTL).

\emph{Fitting} (eller \emph{place-and-route}) oversætter det logiske
kredsløb fra forrige step til enhedsspecifikke logiske elementer og mapper
input/output til brugerdefinerede pins eller automatisk valgte pins hvis de ikke
er definerede.

Når et enhedsspecifik kredsløb er lavet af fitteren kan en \emph{timing analysis} vise om der er tidsmæssige restriktioner i enhedens
logiske elementer, der kan give problemer når programmet kører.

En \emph{simulator} kan bruges til at verificere programmets outputs baseret på
forskellige inputs. En simulering uden timing kan verificere programmets
funktionalitet, imens en simulering med timing kan verificere at programmet
eksekverer problemfrit på en bestemt enhed.

Når alle de forrige steps er gennemkørt kan de enhedsspecifikke
konfigurationsfiler overføres til \emph{device programmeren}, som derefter
programmerer chippen.

\subsubsection{Command line tools}

Alle disse steps og værktøjer beskrevet ovenfor kan køres gennem en grafisk IDE
som f.eks. Quartus II, men er også tilgængelige som command line programmer.
Selvom det er nemt at bruge den grafiske IDE når man prototyper et program, så
gør en command line pipeline det nemt at automatisere hele processen, så man
sikrer at alle steps gennemløbes på samme måde hver gang og det endelige program
altid er verificeret.
