\section{Design og Implementering}

\subsection{Half Adder}

I denne del af øvelsen har vi implementeret en half-adder med de forskellige
kodestilarter: dataflow, structural og behavioural. I \srcref{ex02/HalfAdder.vhd} ses
vores implementering af en enkel HalfAdder entity med tre tilhørende
architectures, en for hver kodestil.

\dsdsrc{ex02/HalfAdder.vhd}{Implementering af half-adder}{tpb}

\subsection{Full Adder}

I denne del af øvelsen har vi implementeret en full-adder med en structural
kodestil. Full-adderen bruger to half-adder entities og en enkelt OR-gate.
Implementeringen ses her i \srcref{ex02/FullAdder.vhd}

\dsdsrc{ex02/FullAdder.vhd}{Implementering af full-adder}{tpb}

\subsection{4-bit ripple carry adder}

I denne del af øvelsen  har vi implementeret en 4-bit adder med structural
kodestil. 4-bit adderen bruger fire full-adder entities med deres carries i
sekvens. Dette skaber en 4-bit ripple carry adder. Implementeringen ses i \srcref{ex02/FourBitAdder.vhd}

\dsdsrc{ex02/FourBitAdder.vhd}{Implementering af 4-bit adder med full-adders}{tpb}

\subsection{Test på boardet}

Ved hjælp af JTAG programmering blev DE-2 boardet flashet og logikken kunne
efterprøves som vist på \figref{sw04.png} og \figref{sw08.png}.
\dsdfig{sw04.png}{6cm}{4 bit adderen viser 8 }{tpb}
\dsdfig{sw08.png}{6cm}{4 bit adderen viser 30 + carry}{tpb}
